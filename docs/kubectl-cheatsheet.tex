\documentclass[a4paper,landscape]{article}
\usepackage[margin=1cm]{geometry}
\usepackage{multicol}
\usepackage{parskip}
\usepackage{enumitem}
\setlist{nosep}

\pagestyle{empty}

\begin{document}

\begin{center}
    {\LARGE \textbf{kubectl Cheat Sheet}}
\end{center}

\begin{multicols}{3}
\section*{Basic Commands}
\begin{itemize}
    \item \texttt{kubectl version} \\
    Show client and server versions.
    \item \texttt{kubectl cluster-info} \\
    Display cluster info.
    \item \texttt{kubectl get nodes} \\
    List all nodes.
    \item \texttt{kubectl get pods} \\
    List all pods in current namespace.
    \item \texttt{kubectl get pods -A} \\
    List all pods in all namespaces.
    \item \texttt{kubectl describe pod <pod>} \\
    Show details of a pod.
\end{itemize}

\section*{Resource Management}
\begin{itemize}
    \item \texttt{kubectl create -f <file>.yaml} \\
    Create resource from file.
    \item \texttt{kubectl apply -f <file>.yaml} \\
    Apply changes from file.
    \item \texttt{kubectl delete -f <file>.yaml} \\
    Delete resource from file.
    \item \texttt{kubectl edit <resource>/<name>} \\
    Edit resource in place.
    \item \texttt{kubectl scale deployment <name> --replicas=3} \\
    Scale deployment.
    \item \texttt{kubectl rollout restart deployment <name>} \\
    Restart a deployment
\end{itemize}

\section*{Logs \& Exec}
\begin{itemize}
    \item \texttt{kubectl logs <pod>} \\
    Show pod logs.
    \item \texttt{kubectl logs -f <pod>} \\
    Follow pod logs.
    \item \texttt{kubectl exec -it <pod> -- /bin/sh} \\
    Exec into pod.
    \item \texttt{kubectl port-forward <pod> 8080:80} \\
    Forward port.
\end{itemize}

\section*{Namespaces}
\begin{itemize}
    \item \texttt{kubectl get ns} \\
    List namespaces.
    \item \texttt{kubectl config set-context --current --namespace=<ns>} \\
    Set default namespace.
\end{itemize}

\section*{Useful Switches}
\begin{itemize}
    \item \texttt{-n <namespace>} \\
    Specify namespace.
    \item \texttt{-o yaml} \\
    Output as YAML.
    \item \texttt{-o json} \\
    Output as JSON.
    \item \texttt{-o wide} \\
    More info in output.
    \item \texttt{-A} \texttt{|} \texttt{--all-namepsaces}\\
    All namespaces.
\end{itemize}

\end{multicols}

\end{document}xw